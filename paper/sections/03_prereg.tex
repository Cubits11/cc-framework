\section{Pre-Registered Hypotheses and Testing Framework}
\label{sec:prereg}

In this section we convert the informal guardrail desiderata---unsafe
complementarity on harmful inputs, bounded benign inflation, and
stability across strata---into precise, finite-sample hypothesis tests
expressed in the dependence coordinates of
Section~\ref{sec:foundations}.
Throughout we use $\lambda_y$ as the canonical ``badness'' coordinate
and $s_y = 1-\lambda_y$ as the corresponding tightness score.

We fix a per-class significance level $\alpha_y$ and use
Theorem~\ref{thm:finite-sample} to obtain confidence radii
$r_y = \varepsilon_y / W_y$ and confidence intervals
\[
  I_y^{(\lambda)} =
  [L_y^{(\lambda)}, U_y^{(\lambda)}]
  := \bigl[\hat\lambda_y - r_y,\; \hat\lambda_y + r_y\bigr] \cap [0,1]
\]
satisfying
$\Pr(\lambda_y \in I_y^{(\lambda)}) \ge 1-\alpha_y$.

\subsection{H1: Unsafe complementarity for AND (harmful class)}

We consider the composed system under an AND rule on harmful inputs
($Y=1$) and write $\lambda_1$ for its dependence coordinate and
$s_1 = 1-\lambda_1$ for its unsafe complementarity score.

We fix a pre-specified threshold $\tau_1 \in (0,1)$ encoding the
maximum acceptable fraction of FH distance from the best-case AND
corner.
Equivalently, the minimum acceptable complementarity score is
$s_1^{\min} = 1-\tau_1$.

\begin{hypothesis}[H1: Unsafe complementarity]
\label{hyp:H1}
\mbox{}
\begin{description}
  \item[Null ($H_{0,1}$):]
    $\lambda_1 > \tau_1$ \quad
    (the harmful-class dependence lies too close to the FH worst case).
  \item[Alternative ($H_{1,1}$):]
    $\lambda_1 \le \tau_1$ \quad
    (the system achieves sufficient unsafe complementarity).
\end{description}
\end{hypothesis}

\paragraph{Test statistic and decision rule.}
Compute $\hat\lambda_1$ and its confidence interval
$I_1^{(\lambda)} = [L_1^{(\lambda)}, U_1^{(\lambda)}]$ as above.
We \emph{reject} $H_{0,1}$ at level $\alpha_1$ whenever
\begin{equation}
  U_1^{(\lambda)} \le \tau_1.
  \label{eq:H1-decision}
\end{equation}
Equivalently, in terms of the complementarity score
$s_1 = 1-\lambda_1$ we reject $H_{0,1}$ whenever the one-sided
$(1-\alpha_1)$ lower confidence bound for $s_1$ lies above
$s_1^{\min} = 1-\tau_1$.

\begin{remark}[Type I control]
Under $H_{0,1}$ we have $\lambda_1 > \tau_1$.
On the event $U_1^{(\lambda)} \le \tau_1$ we must have
$\lambda_1 \notin I_1^{(\lambda)}$, hence
\[
  \Pr_{H_{0,1}}(U_1^{(\lambda)} \le \tau_1)
  \;\le\;
  \Pr_{H_{0,1}}(\lambda_1 \notin I_1^{(\lambda)})
  \;\le\; \alpha_1.
\]
Thus the test~\eqref{eq:H1-decision} has size at most $\alpha_1$.
\end{remark}

\subsection{H2: Benign inflation for OR (benign class)}

For the benign class ($Y=0$) under an OR composition, the dependence
coordinate $\lambda_0$ quantifies benign inflation (distance from the
best FH corner), while $s_0 = 1-\lambda_0$ quantifies benign
preservation.

We fix a pre-specified tolerated inflation level $\tau_0 \in (0,1)$,
or equivalently a preservation score threshold
$s_0^{\min} = 1-\tau_0$.

\begin{hypothesis}[H2: Benign inflation control]
\label{hyp:H2}
\mbox{}
\begin{description}
  \item[Null ($H_{0,0}$):]
    $\lambda_0 > \tau_0$ \quad
    (benign inflation exceeds the allowed FH fraction).
  \item[Alternative ($H_{1,0}$):]
    $\lambda_0 \le \tau_0$ \quad
    (benign inflation is controlled).
\end{description}
\end{hypothesis}

\paragraph{Decision rule.}
Compute $I_0^{(\lambda)} = [L_0^{(\lambda)}, U_0^{(\lambda)}]$ as
before, and reject $H_{0,0}$ at level $\alpha_0$ if
\begin{equation}
  U_0^{(\lambda)} \le \tau_0,
  \label{eq:H2-decision}
\end{equation}
equivalently, if the one-sided $(1-\alpha_0)$ lower confidence bound
for $s_0$ lies above $s_0^{\min}$.
By the same argument as for H1, this test has size at most $\alpha_0$.

\subsection{H3: Strata stability (Simpson guard)}

In many deployments, evaluation data are stratified into groups
(e.g.\ stacks, language families, jurisdictions).
We write $s \in \mathcal{S}$ for a stratum index and
$\lambda_y^{(s)}$ for the dependence coordinate of the composed
system restricted to that stratum.
We also write $\lambda_y^{\mathrm{glob}}$ for the dependence
coordinate computed on the pooled data.

We wish to guard against large latent shifts of dependence across
strata (Simpson effects).
We fix a tolerance $\epsilon_{\mathrm{strata}} > 0$ and require each
stratum to lie within this tolerance of the global system.

\begin{hypothesis}[H3: Strata stability]
\label{hyp:H3}
\mbox{}
\begin{description}
  \item[Null ($H_{0,\mathrm{strata}}$):]
    There exists a stratum $s \in \mathcal{S}$ and a class
    $y \in \{0,1\}$ such that
    \(
      |\lambda_y^{(s)} - \lambda_y^{\mathrm{glob}}|
      > \epsilon_{\mathrm{strata}}.
    \)
  \item[Alternative ($H_{1,\mathrm{strata}}$):]
    For all $s \in \mathcal{S}$ and both classes,
    \(
      |\lambda_y^{(s)} - \lambda_y^{\mathrm{glob}}|
      \le \epsilon_{\mathrm{strata}}.
    \)
\end{description}
\end{hypothesis}

\paragraph{Confidence radii.}
For each class $y$ and stratum $s$, we compute:

\begin{itemize}
  \item an empirical coordinate $\hat\lambda_y^{(s)}$ and radius
    $r_y^{(s)}$ from Theorem~\ref{thm:finite-sample},
  \item the global coordinate $\hat\lambda_y^{\mathrm{glob}}$ and
    radius $r_y^{\mathrm{glob}}$ from the pooled data.
\end{itemize}

We choose per-(class,stratum,global) significance levels
$\alpha_y^{(s)}$ and $\alpha_y^{\mathrm{glob}}$ such that the total
family-wise error across all strata and both classes is bounded by
$\alpha_{\mathrm{strata}}$; for example, a simple Bonferroni choice is
\[
  \alpha_y^{(s)} = \frac{\alpha_{\mathrm{strata}}}{2(|\mathcal{S}|+1)},
  \qquad
  \alpha_y^{\mathrm{glob}} = \frac{\alpha_{\mathrm{strata}}}{2}.
\]

\paragraph{Decision rule (certificate of stability).}
For each class $y$ and each stratum $s \in \mathcal{S}$, consider the
quantity
\begin{equation}
  B_y^{(s)}
  \;:=\;
  |\hat\lambda_y^{(s)} - \hat\lambda_y^{\mathrm{glob}}|
  \;+\; r_y^{(s)} \;+\; r_y^{\mathrm{glob}}.
  \label{eq:simpson-bound}
\end{equation}
By the triangle inequality,
\[
  |\lambda_y^{(s)} - \lambda_y^{\mathrm{glob}}|
  \le
  |\hat\lambda_y^{(s)} - \hat\lambda_y^{\mathrm{glob}}|
  + |\hat\lambda_y^{(s)} - \lambda_y^{(s)}|
  + |\hat\lambda_y^{\mathrm{glob}} - \lambda_y^{\mathrm{glob}}|
  \le B_y^{(s)}
\]
whenever all three coordinates lie in their respective confidence
intervals.

We \emph{reject} $H_{0,\mathrm{strata}}$ at level
$\alpha_{\mathrm{strata}}$ (and thus certify strata stability) if
\begin{equation}
  B_y^{(s)} \le \epsilon_{\mathrm{strata}}
  \quad\text{for all } y \in \{0,1\},\;
  s \in \mathcal{S}.
  \label{eq:H3-decision}
\end{equation}
If condition~\eqref{eq:H3-decision} fails for any $(y,s)$, we are
unable to rule out Simpson-type deviations larger than
$\epsilon_{\mathrm{strata}}$ in that stratum, and we report these
strata explicitly.

\begin{remark}[Conservatism and interpretation]
The rule~\eqref{eq:H3-decision} is conservative: failure to certify
stability does not prove the existence of a Simpson reversal, only
that our finite sample does not suffice to rule one out at the chosen
tolerance and error level.
This is appropriate for safety auditing, where we prefer
non-certification over overconfident claims of homogeneity across
strata.
\end{remark}

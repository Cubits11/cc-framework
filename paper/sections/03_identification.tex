\subsection{Primary safety hypotheses: H1 and H2}
\label{subsec:primary-hypotheses}

Let $p_y = \Pr(Z=1 \mid Y=y)$, $y\in\{0,1\}$, denote the classwise
block rates of a fixed composed rail pair, with Fr\'echet--Hoeffding
bounds $L_y \le p_y \le U_y$.
We define dependence coordinates $\lambda_y$ and tightness scores
$s_y = 1-\lambda_y$ as in Section~\ref{sec:foundations}.

For each class $y$ we choose a critical dependence level
$\tau_y \in (0,1)$ reflecting an operational safety requirement on
$p_y$.
We treat $H_{y,0}:\lambda_y > \tau_y$ as the \emph{unsafe or
undetermined} null, and only reject $H_{y,0}$ when the data support
$\lambda_y \le \tau_y$ with high confidence.

\paragraph{Finite-sample envelopes.}
Given $n_y$ i.i.d.\ samples per class, we form empirical rates
$\hat p_y$ and compute one-sided empirical-Bernstein radii
$\varepsilon_y(n_y,\alpha_y)$ such that
\begin{equation}
  \Pr\bigl( p_y \le \hat p_y + \varepsilon_y \bigr)
  \;\ge\; 1-\alpha_y,
  \qquad y\in\{0,1\}.
  \label{eq:bernstein-H}
\end{equation}
Affine rescaling yields dependence-coordinate envelopes
\[
  \hat\lambda_y =
  \frac{b_y - \hat p_y}{b_y - w_y},\qquad
  U^{(\lambda)}_y =
  \hat\lambda_y + \frac{\varepsilon_y}{b_y - w_y},
\]
which satisfy
$\Pr(\lambda_y \le U^{(\lambda)}_y) \ge 1-\alpha_y$.

\begin{hypothesis}[H1: Complementarity certification (harmful / AND)]
\label{hyp:H1-final}
For the harmful class $y=1$ under an AND composition, fix a
critical dependence level $\tau_1 \in (0,1)$.

\emph{Null (unsafe or undetermined complementarity):}
\[
  H_{1,0}:\quad \lambda_1 > \tau_1.
\]

\emph{Alternative (acceptable complementarity):}
\[
  H_{1,1}:\quad \lambda_1 \le \tau_1.
\]

\emph{Decision rule.}
At per-test level $\alpha_1$ we compute $U^{(\lambda)}_1$ as above
and reject $H_{1,0}$---thereby \emph{certifying complementarity} for
this pair---if and only if
\begin{equation}
  U^{(\lambda)}_1 \le \tau_1.
  \label{eq:H1-rule-final}
\end{equation}
Equivalently, in terms of the tightness score
$s_1 = 1-\lambda_1$, with threshold $s_1^\star = 1-\tau_1$, we
certify whenever the lower confidence limit for $s_1$ exceeds
$s_1^\star$.
\end{hypothesis}

\begin{hypothesis}[H2: Benign-inflation certification (benign / OR)]
\label{hyp:H2-final}
For the benign class $y=0$ under an OR composition, fix a critical
dependence level $\tau_0 \in (0,1)$.

\emph{Null (excess or undetermined benign inflation):}
\[
  H_{2,0}:\quad \lambda_0 > \tau_0.
\]

\emph{Alternative (acceptable benign inflation):}
\[
  H_{2,1}:\quad \lambda_0 \le \tau_0.
\]

\emph{Decision rule.}
At per-test level $\alpha_0$ we compute $U^{(\lambda)}_0$ and reject
$H_{2,0}$---\emph{certifying acceptable benign inflation}---if and
only if
\begin{equation}
  U^{(\lambda)}_0 \le \tau_0.
  \label{eq:H2-rule-final}
\end{equation}
Equivalently, with benign-preservation score $s_0=1-\lambda_0$ and
threshold $s_0^\star = 1-\tau_0$, we certify whenever the lower
confidence limit for $s_0$ lies above $s_0^\star$.
\end{hypothesis}

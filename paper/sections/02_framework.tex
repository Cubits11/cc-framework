\section{The CC-Framework}
\label{sec:framework}

\subsection{Guardrail Model}

\begin{definition}[Guardrail]
\label{def:guardrail}
A guardrail $R$ is a binary classifier mapping inputs to accept/reject decisions:
\[
R: \mathcal{X} \to \{0, 1\}
\]
where $R(x) = 1$ indicates the input is allowed (passed) and $R(x) = 0$ indicates it is blocked (rejected).
\end{definition}

\begin{definition}[Two-World Protocol]
\label{def:two-world}
Define two distributions over inputs:
\begin{itemize}
    \item $\mathcal{D}_0$: benign/legitimate user inputs
    \item $\mathcal{D}_1$: adversarial/harmful inputs (threats)
\end{itemize}
A guardrail is evaluated on test sets $S_0 \sim \mathcal{D}_0$ and $S_1 \sim \mathcal{D}_1$ to measure:
\begin{align*}
\text{TPR}(R) &= \Pr_{x \sim \mathcal{D}_1}[R(x) = 0] \quad \text{(threats blocked)} \\
\text{FPR}(R) &= \Pr_{x \sim \mathcal{D}_0}[R(x) = 0] \quad \text{(benign blocked)}
\end{align*}
\end{definition}

\begin{definition}[Youden's $J$ Statistic]
\label{def:youden}
For a guardrail $R$, Youden's $J$ is:
\[
J(R) = \text{TPR}(R) - \text{FPR}(R)
\]
Range: $J \in [-1, 1]$ where:
\begin{itemize}
    \item $J = 1$: perfect classifier (all threats blocked, no benign blocked)
    \item $J = 0$: random classifier
    \item $J = -1$: adversarial classifier (inverted behavior)
\end{itemize}
\end{definition}

\subsection{Composition}

\begin{definition}[Boolean Composition]
\label{def:composition}
For guardrails $R_1, \ldots, R_n$, a composition rule is a Boolean function:
\[
g: \{0,1\}^n \to \{0,1\}
\]
The composed system outputs:
\[
Z = g(R_1, \ldots, R_n)
\]
Common rules:
\begin{itemize}
    \item AND: $g_\land(r_1, \ldots, r_n) = \min(r_1, \ldots, r_n)$ (all must pass)
    \item OR: $g_\lor(r_1, \ldots, r_n) = \max(r_1, \ldots, r_n)$ (any can pass)
\end{itemize}
\end{definition}

\begin{definition}[Independence Baseline]
\label{def:independence}
Under the assumption that $R_1$ and $R_2$ make independent errors, the composed Youden's $J$ satisfies:
\[
J_{\text{ind}}(R_1 \land R_2) = J(R_1) + J(R_2) - J(R_1) \cdot J(R_2).
\]
We define the additive deviation as
\[
\Delta_{\text{add}} = J_{\text{actual}} - J_{\text{ind}},
\]
which measures synergy (positive) or interference (negative).
\end{definition}

\subsection{The CC Metric}

\begin{definition}[Compositional Coupling Coefficient]
\label{def:cc-max}
For a composition of guardrails, the compositional coupling coefficient is:
\[
\mathrm{CC}_{\max} = \frac{J_{\text{actual}}}{\max\{J(R_1), J(R_2)\}}
\]
where $J_{\text{actual}}$ is the measured Youden's $J$ of the composed system.
\end{definition}

\subsection{Regime Taxonomy}

\begin{definition}[Dependence Regimes]
\label{def:regimes}
Based on $\mathrm{CC}_{\max}$, we classify compositions into three regimes:
\begin{itemize}
    \item \textbf{Constructive} ($\mathrm{CC}_{\max} < 0.95$): composition improves beyond the best single guardrail.
    \item \textbf{Independent} ($0.95 \le \mathrm{CC}_{\max} \le 1.05$): composition matches the best single guardrail within a tolerance band.
    \item \textbf{Destructive} ($\mathrm{CC}_{\max} > 1.05$): composition underperforms relative to the best single guardrail.
\end{itemize}
The $\pm 5\%$ band is a conservative policy margin to account for sampling variability.
\end{definition}

\subsection{Unknown Dependence Analysis}

When the joint distribution of guardrail errors is unknown, we employ Fr\'echet--Hoeffding bounds to characterize worst-case and best-case dependence.

\begin{theorem}[Fr\'echet--Hoeffding Envelope]
\label{thm:fh-bounds}
For random variables $X_1, X_2$ with marginal CDFs $F_1, F_2$, their joint CDF $F$ satisfies:
\[
F^-(x_1, x_2) \le F(x_1, x_2) \le F^+(x_1, x_2)
\]
where:
\begin{align*}
F^-(x_1, x_2) &= \max(0, F_1(x_1) + F_2(x_2) - 1) \quad \text{(countermonotonic)}, \\
F^+(x_1, x_2) &= \min(F_1(x_1), F_2(x_2)) \quad \text{(comonotonic)}.
\end{align*}
\end{theorem}

\begin{definition}[Regime Uncertainty]
\label{def:uncertainty}
A composition exhibits \textbf{regime uncertainty} if the interval
$[\mathrm{CC}^-_{\max}, \mathrm{CC}^+_{\max}]$ computed via FH bounds overlaps the independence band $[0.95, 1.05]$.
In such cases, the true regime cannot be determined without additional information about the dependence structure.
\end{definition}

\begin{remark}
This framework explicitly acknowledges epistemic uncertainty: when dependence is unmeasured, we bound the range of possible outcomes rather than assuming independence by default.
\end{remark}

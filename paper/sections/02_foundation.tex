\documentclass[11pt]{article}

\usepackage{amsmath,amssymb,amsthm}
\usepackage[margin=1in]{geometry}

\newtheorem{definition}{Definition}
\newtheorem{remark}{Remark}

\begin{document}

\section{Dependence Coordinates and Tightness Scores}

We consider a binary class label $Y \in \{0,1\}$, with $Y=1$
denoting harmful inputs and $Y=0$ benign inputs.
Let $Z \in \{0,1\}$ be the output of a composed guardrail system
(e.g.\ an AND/OR rule over individual rails), where $Z=1$ means
``blocked''.
For each class $y \in \{0,1\}$ we write
\[
  p_y = \Pr(Z = 1 \mid Y = y)
\]
for the classwise block rates.

Given only per-rail marginal robustness at some operating point, the
joint dependence structure is unknown and $p_y$ is constrained to lie
in a Fr\'echet--Hoeffding (FH) interval
\[
  L_y \;\le\; p_y \;\le\; U_y, \qquad y \in \{0,1\}.
\]

\subsection{Best and worst FH extremes}

For safety, we orient these bounds differently by class:

\begin{definition}[Safety orientation]
For the harmful class $y=1$, higher block rate is better, so we
define
\[
  b_1 = U_1 \quad \text{(best case)}, \qquad
  w_1 = L_1 \quad \text{(worst case)}.
\]
For the benign class $y=0$, lower block rate is better, so we define
\[
  b_0 = L_0 \quad \text{(best case)}, \qquad
  w_0 = U_0 \quad \text{(worst case)}.
\]
We also define the FH width
\[
  W_y = b_y - w_y > 0, \qquad y \in \{0,1\}.
\]
\end{definition}

Thus $b_y$ and $w_y$ are always ``best'' and ``worst'' in terms of
safety, even though they correspond to different endpoints of the
FH interval for $y=1$ and $y=0$.

\subsection{Canonical dependence coordinates}

We now define coordinates that locate the system inside the FH box.

\begin{definition}[Dependence coordinates]
On non-degenerate FH intervals, we define the
\emph{dependence coordinates} $\lambda_y \in [0,1]$ by
\begin{equation}
  \lambda_y \;=\; \frac{b_y - p_y}{b_y - w_y},
  \qquad y \in \{0,1\}.
  \label{eq:lambda-def}
\end{equation}
Equivalently,
\[
  p_y \;=\; (1-\lambda_y)\,b_y + \lambda_y\,w_y.
\]
We interpret $\lambda_y = 0$ as ``best-case FH performance'' and
$\lambda_y = 1$ as ``worst-case FH performance'' for class $y$.
When $b_y = w_y$ the FH interval collapses and $\lambda_y$ is
degenerate and not identifiable from $p_y$.
\end{definition}

Concretely, with the safety orientation above:
\begin{align*}
  \lambda_1 &= \frac{U_1 - p_1}{U_1 - L_1}, &
  p_1 &= (1-\lambda_1)U_1 + \lambda_1 L_1, \\
  \lambda_0 &= \frac{p_0 - L_0}{U_0 - L_0}, &
  p_0 &= (1-\lambda_0)L_0 + \lambda_0 U_0.
\end{align*}

Thus $\lambda_y$ is always a ``badness'' coordinate: small values
mean the system sits near the best FH extreme for class $y$, large
values mean it sits near the worst extreme.

\subsection{Tightness scores (goodness view)}

For interpretability and dashboards it is often convenient to work
with scores that run in the intuitive direction ``higher is better''.

\begin{definition}[Tightness scores]
The \emph{FH tightness scores} $s_y \in [0,1]$ are defined as
\begin{equation}
  s_y \;=\; 1 - \lambda_y
          \;=\; \frac{p_y - w_y}{b_y - w_y},
  \qquad y \in \{0,1\}.
  \label{eq:s-def}
\end{equation}
Thus $s_y = 1$ means the system attains the best FH extreme $b_y$,
and $s_y = 0$ means it attains the worst extreme $w_y$.
\end{definition}

Instantiating \eqref{eq:s-def} with the safety orientation gives
\begin{align*}
  s_1 &= \frac{p_1 - L_1}{U_1 - L_1}
       &&\text{(unsafe complementarity / redundancy score)},\\[0.5ex]
  s_0 &= \frac{U_0 - p_0}{U_0 - L_0}
       &&\text{(benign preservation score)}.
\end{align*}

\begin{remark}[Relation to earlier $g$-notation]
In previous drafts we used classwise scores
\[
  g_1 = \frac{p_1 - L_1}{U_1 - L_1}, \qquad
  g_0 = \frac{p_0 - L_0}{U_0 - L_0}.
\]
These relate to the new coordinates as
\[
  g_1 = s_1 = 1 - \lambda_1, \qquad
  g_0 = \lambda_0 = 1 - s_0.
\]
For the remainder of the paper we treat $\lambda_y$ as the canonical
dependence coordinates (used in all theorems, derivatives, and
stress bounds) and $s_y$ as the preferred human-facing scores.
\end{remark}

Empirical estimators $\hat p_y$ yield empirical coordinates
$\hat\lambda_y$ and scores $\hat s_y$ via the same formulas, simply
by substituting $\hat p_y$ for $p_y$.
Finite-sample confidence intervals for $p_y$ induce intervals for
$\lambda_y$ and $s_y$ via affine rescaling; see
Section~\textsection2 for details.

\end{document}
